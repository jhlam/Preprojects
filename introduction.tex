\chapter{Introduction} \label{intro}

Using graphs as representation of real world network and graph traversal algorithms to iterate over graphs are seen in different scientific domains \cite{HybridBFS2015}. Graph simulations have been used to study how a disease would spread through a social network\cite{DiseasOutbreak}, predict how efficient viral marketing could be \cite{ViralMarketing} and how a new trend would spread through a community \cite{MaximizeSpread2003}. Seeing how data propagates through a network is known as information diffusion. We can compare the diffusion as a company trying to promote a new product to the public, the public would be the network and each node can either adopt the new trend, or ignore it. One important aspect of information diffusion is the starter set, also known as $seed$ $nodes$. A seed node is the initial starter nodes that is activated, much like a person reciving a free sample from a company trying to promote a new product. By giving someone a free sample, it can help promote the product, but choosing which person to give the sample to is essential. Choosing someone with few friends and few connection would be a waste of product, giving too much would hurt the companys profit and giving too little would result in not enough exposure.

\subsection{The problem}
To select the best seed nodes is an NP-hard problem known as influential maxoimization problem \cite{MaximizeSpread2003} and one of the solution is the greedy algorithm \cite{MaximizeSpread2015} proposed by Kempe et al. The greedy algorithm goes through all the nodes and computes the effect on the network that the nodes had. The greedy algorithm takes the $\it{k}$ top most influential nodes as the seed nodes. When a new seed node is to be added to the starter set $\it{S}$, the algorithm iterate over all the nodes in the network that is not in $\it{S}$ and compute the spread. This project is a collaboration with the Arctic Green Computing Group. 

\subsection{Assignment text}
"Information diffusion is a field of network research where a message, starting at a set of seed nodes, is propagated through the edges in a graph according to a simple model. Simulations are used to measure the coverage and speed of the diffusion and are useful in modeling a variety of phenomena such as the spread of disease, memes on the Internet, viral marketing and emergency messages in disaster scenarios.

The effectiveness of a given spreading model is dependent on the initially infected nodes, or seeds. Seed selection for an optimal spread is an NP hard problem and is normally approximated by selecting high-degree nodes or using heuristic methods such as discount-degree or choosing nodes at different levels of the k-core.

This project will explore the feasibility of hardware accelerated seed selection in large graphs as a variation of breadth-first search (BFS) where the decision to visit and infect a child node relies on the outcome of a coin flip. The student is expected to conduct a thorough review of the literature on seed selection algorithms, diffusion models, with emphasis on parallelisation and hardware acceleration and cache models to reduce memory bottlenecks. Hardware design and simulations are possible if time permits."


\subsection{Report setup}
I have chosen to interpret the task as to understand the fundamental concepts of graph theory, the core concept of information diffusion and seed selection. In this report, we will to focus on one specific model of information diffusion which is the independent cascade model, which will be further discussed in Section \ref{background}. Different solutions that might be able to improve the independent cascade model and the seed selection were studied. Solution as independent cascade model as sparse matrix-vector multiplication on boolean semirings, different parallalization of BFS and cache oblivious sparce matrix multiplication will be discussed in chapter \ref{discussion} and chapter \ref{futureWork}. In this report, I have performed a similar experiment as the experiment done in \cite{MaximizeSpread2003} with the greedy algorithm, the high degree algorithm, the random algorithm and a modified version of the greedy algorithm we are calling independent greedy algorithm in chapter \ref{methode}. The result from the simulations are presented in chapter \ref{results}. And in chapter \ref{discussion}, we will explain some of the results we got. We will go through some of the notation commonly used in graph theory in Chapter \ref{background} and we will look at how we can perform standard graph algorithm such as BFS as matrix-vector multiplication. We will propose some idea that could potentially improve the seed selection algorithm, the independent cascade model or the diffusion of information.
