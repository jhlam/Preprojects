\chapter{Introduction} 

Using graphs as representation of real world network and graph traversal algorithms to itterate over graphs are seen in different scientific domains\cite{HybridBFS2015}. Graph simulation have been used to simulate how a disease would spread through a social network\cite{DiseasOutbreak}, predict how efficient viral marketing is and how a new trend would spread through a community\cite{MaximizeSpread2003}. Seeing how some data propagates through a network is known as information diffusion. We can compare the diffusion as a company trying to promote a new product to the public, the public would be the network and each node can either adopt the new trend, or ignore it. One important aspect of information diffusion is the starter set, or the seed nodes. The seed node is the inital starter nodes that are activated, much like a person reciving a free sample from a company trying to promote a new product. By giving someone such a free product it can help promote the product, but choosing which person to give the smaple to is important. Choosing someone with few friends and few connection would be a waste of product, giving too much would hurt the companys profit and bit giving would result in not enough exposure.

To select the best seed noodes is an NP-hard problem\cite{MaximizeSpread2003}. One of the solution is the greedy algorithm\cite{MaximizeSpread2015} proposed by Kempe et al. The greedy algorithm goes through all the nodes and computes the effect on the network that nodes had. The greedy algorithm takes the $\it{k}$ top most influential nodes as the starter node. When a new seed node is to be added to the starter set $\it{S}$, the algorithm itterate over all the node in the network that is not in $\it{S}$ and compute the spread.

This project is a coaboration with the Arctic Green Computing Group. The task assigned was:
"Information diffusion is a field of network research where a message, starting at a set of seed nodes, is propagated through the edges in a graph according to a simple model. Simulations are used to measure the coverage and speed of the diffusion and are useful in modelling a variety of phenomena such as the spread of disease, memes on the Internet, viral marketing and emergency messages in disaster scenarios.

The effectiveness of a given spreading model is dependent on the initially infected nodes, or seeds. Seed selection for an optimal spread is an NP hard problem and is normally approximated by selecting high-degree nodes or using heuristic methods such as discount-degree or choosing nodes at different levels of the k-core.

This project will explore the feasibility of hardware accelerated seed selection in large graphs as a variation of breadth-first search (BFS) where the decision to visit and infect a child node relies on the outcome of a coin flip. The student is expected to conduct a thorough review of the literature on seed selection algorithms, diffusion models, with emphasis on parallelisation and hardware acceleration and cache models to reduce memory bottlenecks. Hardware design and simulations are possible if time permits."

I have choosen to interpent the task as to understand the fundamental concepts of graph theory, the core concept of information diffusion and seed selection. I have choosen to focus on one specific model of information diffusion which is the independent cascade model, which will be furthur disscussed in seciton[BACKGROUND]. I have been looking at different proposed solution that might be able to improve the independent cascade model and the seed selection.

I have in this report performed a simillar experiment as the experiment done in \cite{Maximizespread2003} with the greedy algorithm, the high degree algorithm, the random algorithm and a modified version of the greedy algorithm we are calling independent greedy algorithm in [METHODE]. We will go through some of the notation commonly used in graph theory in[Background] and we will look at how we can performe standard graph algorithm such as BFS as matrix-vector multiplication. We will propose some idea that chould potentially improve the seed selection algorithm, the independent cascade model or the diffusion of information. 
