\chapter{Background} \lable{chap_background}

In this chapter, we will look at the fundamental concepts and background information needed to understand the the different infections model, problem with seed-selection for social graphs, and solving Breadth-first search using matrix multiplication.

\section{Network}
Network is a collection of nodes with edges connecting them together\cite{TheStrucandcomplex}. There are different types of network or " {\it graph}. These include from {\it social graph}, {\it small-world graph} et al. Each of them have a different property 

A social network is a set of people connected to each other via some form of contact or interactions. A social network often display information regarding connection, interaction or location of a set of people\citeP{Tdt22Paper5}. The social network forms patterns regarding friendships, buissness interactions between companies and familie history/ ancestral tree. One properties that is common in social networks is clusters of communities inside the network, or {\it undergroups}. {\it Undergroups} are clusters of people that creates a closed hub of people thats connected to each other. Different hubs are connected to each other via small nuber of \" Bridges\" \cite{MaxSpread}.  


\section{Data diffusion}
Datta diffusion is looking at how information is propagated through a network or a graph. An example would be how a new internett meme, a new trend or how a new disease is spread through a community. The process consist of a set of starter nodes that are \"infected\", during each timestep, there are a chance that the \"infected\" node would \"infect\" its neighbour. To start such a simulation, we would need to first pick out a set of inital \"starter node\". These {\it k} starter node is a set of node that in the initial time-stepp is infected. They will pass on the informatin/infection during each timestep and the information/infection will propagate through the network.

\section{Basic Diffusion Models}
When we talk about information propagation, we can look at how medical and technological innovations is spread through a social network. We can simulate those kind of behaviour with different diffusion models. There are two basic diffusion models used to simulate the propagation of informaiton through a network\cite{kempe}, the {\it linear threshold model} and the {\it independent cascade model}\cite{kempe}.

This process can be simulated by looking at a social network and how information propagetes through the network. A social network is a graph of relationship and interactions within a group of individuals. 
We look at each node in the graph as a person, they can be either active, or inactive. The activation of a node depends on which diffusion model we choose.

The linear threshold model uses a threshold $\theta_v$ bewteen the interval [0,1], which represent the faction of {\it v}'s neighbors that need to be active to activate node{\it v}. The Linear Threshold Model activates the current node {\it v} when the weight{\it b_{v,w}} $> \theta$ from its neighbour {\it w} outweights the $\theta_v$. This is more in line with the situation were each person have a chance to adopt to a new trend if exposed to the trend enough time by his close friends. An examlpe would be a product promoted on social network like twitter and facebook. The user will adopt the new trend if he is exposed to it from eough friends or idols. 

The independent cascade model changes states with a probability $p_{\it v, w}$, where {\it v }is current node, and {\it w} is it's neighbor. During each propagation, the node {\it v} have a  $p_{\it v, w}$ chance to change state if the neighbour {\it w} changed state. If during timestep {\it t} a node {\it v} changed state, its neighbour {\it w} would have a  $p_{\it v, w}$ chance to change state in the next time stepp. An exaple here would be spread of a disease. The current node {\it v} have a chance($\theta_v$) to infect its neighbour

\section{Breadth First Search as a matrix multiplication.}
By looking at the 	connectivity matrix, we can se that a connectivity matrix is a graph represented in a matrix format. The BFS can be achived by looking at the BFS operations as a matrix multiplication \cite{algoToMath}. If we look at the graph as a connectivity matrix, we can apply matrixe multiplication to generate the breadth first search. 

\section{SIR/SIS}
There are two different epidemic model that we will be looking at, the {\it SIR} ( susceptible, infected, removed,) and the SIS mode(susceotible, infected, susceptible). The SIR model stands for susceptible, infected and removed. The node would have three different states, the susceptible state mean that the node is susceptible to the disease, or in this example, change state. The infected state is the state were the node {\it v} is infected. The state Removed/Recovered, is the state were the node {\it v} have the diseas removed, or have recover from the infection. In this model, the Removed/Recovered state is not susvveptible to the infection again. 

The SIS stands for susceptible, infecded and susceptible again, this model works if the infected node is susceptible to the infection again. 