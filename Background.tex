\chapter{Background} \lable{chap_background}

In this chapter, we will look at the fundamental concepts and background information needed to understand the the different infections model, problem with seed-selection for social networks, and solving Breadth-first search using matrix multiplication. This chapter will contain notations that we would use throughout the report and used in the field. 

\section{Network}
A {\it network} is a collection of {\it Vertices}, commonly known as nodes with \{it edges} connecting them together\cite{ComplexNetwork2003}. The edges serves as a connection, or a \"bridge\" between the nodes, while the nodes can reprecent something, or containing information. In the real world, multiple systems takes the form of networks around the world, examples are the internettt, the World Wide Web, social media like Facebook, twitter etc.  There are different types of network or " {\it graphs}. These include from {\it social network}, {\it sinformation networks}, {\it technological networks} ,and {\it biological network}.  Each of them have a different properties, but we will focus more about social network.

A social network is a set of people connected to each other via some form of contact or interactions\cite{ComplexNetwork2003}. The nodes are people while the edges are the connections between peoples. The social network display information regarding connection, interaction or location of a set of people.It forms patterns regarding friendships, business interactions between companies and families history/ ancestral tree. The social network is often used in social science\cite{ComplexNetwork2003}. Some noteable experiments are \cite{smallWorldExperiment}, which looks at the small world problem. The small world problem can be summarized as: \"what is the probability that any two people, selected arbitrarily from a large population, such as that of the United States, will know each other?\"\cite{SmallworldExperiment1969}. This in itselfe is not that interesting, the \cite{SmallworldExperiment1969} takes that altho person {\it a} and {\it z} does not know each other, do they have a set of individuals \{{\it b_1, b_2, ... b_n}\} who are mutual friend or even a \"chain\" of such individual({\it a-b-c-...-y.z}).

Some properties that networks often exhibits are the small world effect, Transitivity or clustering, Degree correlations, and Community structure\cite{ComplexNetwork2003}.

\subsection{The small world effect}
The small world effect shows that in most graph, most pair of verices seems to be connected to each other with a short(no more than 6) path\cite{ComplexNetwork2003}. The small effect shows us that the geodesic distance from node {\it a} to {\it b} is no more than 6. a graph including the small world effect is called a {\it small-world graph}. A small-world graph have a diameter no more than 6.\citeP{HybridBFS2015}. 

\subsection{Transistivity/clustering}
Transistivity, also sometimes called clustering is the propertie that there are a 

\subsection{Degree correlation}

\subsection{community structure}



\section{Data diffusion}
Data diffusion is looking at how information is propagated through a network or a graph. An example would be how a new Internet meme, a new trend or how a new disease is spread through a community. The process consist of a set of starter nodes that are \"infected\", during each time-step, there are a chance that the \"infected\" node would \"infect\" its neighbor. To start such a simulation, we would need to first pick out a set of initial \"starter node\". These {\it k} starter node is a set of node that in the initial time-step is infected. They will pass on the information/infection during each time-step and the information/infection will propagate through the network.

\section{Basic Diffusion Models}
When we talk about information propagation, we can look at how medical and technological innovations is spread through a social network. We can simulate those kind of behavior with different diffusion models. There are two basic diffusion models used to simulate the propagation of information through a network\cite{MaximizeSpread2003}, the {\it linear threshold model} and the {\it independent cascade model}\cite{MaximizeSpread2003}.

This process can be simulated by looking at a social network and how information propagates through the network. We look at each node in the graph as a person, they can be either active, or inactive. The activation of a node depends on which diffusion model we choose. A active node is \"infected\", while the inactive is the \"healthy\" ones. The activation of each nodes is dependent on which model we pick.

The linear threshold model uses a threshold $\theta_v$ between the interval [0,1], which represent the faction of {\it v}'s neighbors that need to be active to activate node{\it v}. The Linear Threshold Model activates the current node {\it v} when the weight{\it b_{v,w}} $> \theta$ from its neighbor {\it w} outweighs the $\theta_v$. This is more in line with the situation were each person have a chance to adopt to a new trend if exposed to the trend enough time by his close friends. An example would be a product promoted on social network like twitter and Facebook. The user will adopt the new trend if he is exposed to it from enough friends or idols. 

The independent cascade model changes states with a probability $p_{\it v, w}$, where {\it v }is current node, and {\it w} is it's neighbor. During each propagation, the node {\it v} have a  $p_{\it v, w}$ chance to change state if the neighbor {\it w} changed state. If during time-step {\it t} a node {\it v} changed state, its neighbor {\it w} would have a  $p_{\it v, w}$ chance to change state in the next time step. An example here would be spread of a disease. The current node {\it v} have a chance($\theta_v$) to infect its neighbor

\section{Breadth First Search as a matrix multiplication.}
By looking at the 	connectivity matrix, we can see that a connectivity matrix is a graph represented in a matrix format. The BFS can be achieved by looking at the BFS operations as a matrix multiplication \cite{algoToMath}. If we look at the graph as a connectivity matrix, we can apply matrix multiplication to generate the breadth first search. 

\section{SIR/SIS}
There are two different epidemic model that we will be looking at, the {\it SIR} ( susceptible, infected, removed,) and the SIS mode(susceptible, infected, susceptible).Both is used to simulate how a disease or an epidemic can spread through the general population, in this case, we can use this 

The SIR model stands for susceptible, infected and removed. The node would have three different states, the susceptible state mean that the node is susceptible to the disease, or in this example, change state. The infected state is the state were the node {\it v} is infected. The state Removed/Recovered, is the state were the node {\it v} have the disease removed, or have recover from the infection. In this model, the Removed/Recovered state is not susceptible to the infection again. 

The SIS stands for susceptible, infected and susceptible. Unlike the SIR this model can reinfect the recovered nodes. This model is used for simulation of outbreak of disease. 