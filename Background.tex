\chapter{Background} \lable{chap_background}

In this chapter, we will look at the fundamental concepts and background information needed to understand the the different infections model, problem with seed-selection for social graphs, and solving Breadth-first search using matrix multiplication.

\section{Basic Diffusion Models}
When we talk about information propagation, we can look at how medical and technological innovations is spread through a social network. We can simulate those kind of behaviour with diffusion models. There are two basic diffusion models used to simulate the propagation of informaiton through a network\cite{kempe}, the {\it linear threshold model} and the {\it independent cascade model}\cite{kempe}.

This process can be simulated by looking at a social-graph and how information propagetes through the network. A social network is a graph of relationship and interactions within a group of individuals. 
We look at each node in the graph as a person, they can be either active, or inactive. The activation of a node depends on which diffusion model we choose.

The linear threshold model uses a threshold $\theta_v$ bewteen the interval [0,1], which represent the faction of {\it v}'s neighbors that need to be active to activate node{\it v}. The Linear Threshold Model activates the current node {\it v} when the weight{\it b_{v,w}} $> \theta$ from its neighbour {\it w} outweights the $\theta_v$. 

The independent cascade model changes states with a probability $p_{\it v, w}$, where {\it v }is current node, and {\it w} is it's neighbor. During each propagation, the node {\it v} have a  $p_{\it v, w}$ chance to change state if the neighbour {\it w} changed state. If during timestep {\it t} a node {\it v} changed state, its neighbour {\it w} would have a  $p_{\it v, w}$ chance to change state in the next time stepp.