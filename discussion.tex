\chapter{Discussion}
The result was somewhat surprising, the greedy algorithm that I thought was gonna exel in this simulation was not a huge improvement over the other algorithm. The random algorithm performed better then i expected and the degree algorithm was in some stages better then the greedy algorithm. The independent coverage algorithm performed better or equally good as most of the other algorithm. Toward the end where we choose twenty seed nodes, most of the algorithm performed somewhat the same.

These results can be explained by looking at our graph. The graph we ran the simulation over was a small graph compared to real world graphs, by choosing at most 20 seed nodes, a large percentage of the graph would be covered. Since the graphs was all connected in a network, the small world effect would result in that every vertex would be subjected to activation. Our global percentage was 0.05. 

One paper showed how implementing a palatalized BFS algorithm on a distributed system reduced the communication times by a factor of 3.5, compared to a common vertex based approach\cite{ParallelBFS2011}.  In scientific application, ,graph based computation is pervasive and often data-intensive, this especially for distributed systems. 

ISNERT INFORMATION 


It would be interesting to see how the palatalized BFS would improve the computation time for our simulation.