\chapter{Discussion}
One paper showed how implementing a parallalized bfs algorithm on a distributed system reduced the communication times by a factor of 3.5, compared to a common vertex based approach\cite{ParallelBFS2011}.  In scientific application, ,graph based computation is pervasived and often data-intensive, this especially for distributed systems. 

ISNERT INFORMATION 


It would be interesting to see how the parallelized BFS would improve the computation time for our simulation.